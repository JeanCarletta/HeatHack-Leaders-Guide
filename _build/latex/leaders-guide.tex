%% Generated by Sphinx.
\def\sphinxdocclass{jupyterBook}
\documentclass[letterpaper,10pt,english]{jupyterBook}
\ifdefined\pdfpxdimen
   \let\sphinxpxdimen\pdfpxdimen\else\newdimen\sphinxpxdimen
\fi \sphinxpxdimen=.75bp\relax
\ifdefined\pdfimageresolution
    \pdfimageresolution= \numexpr \dimexpr1in\relax/\sphinxpxdimen\relax
\fi
%% let collapsible pdf bookmarks panel have high depth per default
\PassOptionsToPackage{bookmarksdepth=5}{hyperref}
%% turn off hyperref patch of \index as sphinx.xdy xindy module takes care of
%% suitable \hyperpage mark-up, working around hyperref-xindy incompatibility
\PassOptionsToPackage{hyperindex=false}{hyperref}
%% memoir class requires extra handling
\makeatletter\@ifclassloaded{memoir}
{\ifdefined\memhyperindexfalse\memhyperindexfalse\fi}{}\makeatother

\PassOptionsToPackage{warn}{textcomp}

\catcode`^^^^00a0\active\protected\def^^^^00a0{\leavevmode\nobreak\ }
\usepackage{cmap}
\usepackage{fontspec}
\defaultfontfeatures[\rmfamily,\sffamily,\ttfamily]{}
\usepackage{amsmath,amssymb,amstext}
\usepackage{polyglossia}
\setmainlanguage{english}



\setmainfont{FreeSerif}[
  Extension      = .otf,
  UprightFont    = *,
  ItalicFont     = *Italic,
  BoldFont       = *Bold,
  BoldItalicFont = *BoldItalic
]
\setsansfont{FreeSans}[
  Extension      = .otf,
  UprightFont    = *,
  ItalicFont     = *Oblique,
  BoldFont       = *Bold,
  BoldItalicFont = *BoldOblique,
]
\setmonofont{FreeMono}[
  Extension      = .otf,
  UprightFont    = *,
  ItalicFont     = *Oblique,
  BoldFont       = *Bold,
  BoldItalicFont = *BoldOblique,
]



\usepackage[Bjarne]{fncychap}
\usepackage[,numfigreset=1,mathnumfig]{sphinx}

\fvset{fontsize=\small}
\usepackage{geometry}


% Include hyperref last.
\usepackage{hyperref}
% Fix anchor placement for figures with captions.
\usepackage{hypcap}% it must be loaded after hyperref.
% Set up styles of URL: it should be placed after hyperref.
\urlstyle{same}


\usepackage{sphinxmessages}



        % Start of preamble defined in sphinx-jupyterbook-latex %
         \usepackage[Latin,Greek]{ucharclasses}
        \usepackage{unicode-math}
        % fixing title of the toc
        \addto\captionsenglish{\renewcommand{\contentsname}{Contents}}
        \hypersetup{
            pdfencoding=auto,
            psdextra
        }
        % End of preamble defined in sphinx-jupyterbook-latex %
        

\title{Leaders' Guide}
\date{Sep 13, 2022}
\release{}
\author{Jean Carletta}
\newcommand{\sphinxlogo}{\vbox{}}
\renewcommand{\releasename}{}
\makeindex
\begin{document}

\pagestyle{empty}
\sphinxmaketitle
\pagestyle{plain}
\sphinxtableofcontents
\pagestyle{normal}
\phantomsection\label{\detokenize{intro::doc}}


\sphinxAtStartPar
This document is a guide for the group leader and the engineer to use in running group sessions.  It gives an introduction to what sessions are like, and then detailed plans for each session that you can follow.

\sphinxstepscope


\chapter{The group process}
\label{\detokenize{group-process:the-group-process}}\label{\detokenize{group-process::doc}}
\sphinxAtStartPar
:TODO: Pam to correct this text and add further reading if A Recipe comes from somewhere else

\sphinxAtStartPar
Your groups may end up talking about some difficult and contentious issues that relate deeply to the members’ values and sense of who they are and what their community is for.  In this kind of group, it’s important to avoid creating unresolved discord.  The goal is for the members to establish common ground, acknowledge their differences,  and come to a common sense of how to proceed for the good of the community as a whole.

\sphinxAtStartPar
To aid this process, we use “A Recipe” as a basic set of ground rules. We’ve put these in the wider reading materials, but they’re worth repeating here:
\begin{itemize}
\item {} 
\sphinxAtStartPar
\sphinxstylestrong{A} — Avoid assumptions

\item {} 
\sphinxAtStartPar
\sphinxstylestrong{R} — Respect each other

\item {} 
\sphinxAtStartPar
\sphinxstylestrong{E} — Everyone’s contribution counts

\item {} 
\sphinxAtStartPar
\sphinxstylestrong{C} — Confidentiality

\item {} 
\sphinxAtStartPar
\sphinxstylestrong{I} — I speak for myself

\item {} 
\sphinxAtStartPar
\sphinxstylestrong{P} — Pass if you have nothing you wish to say

\item {} 
\sphinxAtStartPar
\sphinxstylestrong{E} — Expect to participate

\end{itemize}

\sphinxAtStartPar
It can be helpful to write these out large to stick where everyone can remind themselves of them during the sessions.

\sphinxstepscope


\chapter{The Structure of a Session}
\label{\detokenize{what-sessions-are-like:the-structure-of-a-session}}\label{\detokenize{what-sessions-are-like::doc}}
\sphinxAtStartPar
For each session, we have prepared a summary sheet that will tell you what to do when and guide timings, so that you can see everything at a glance.  You can find these summary sheets :TODO: INSERT LINK.  The summary sheet shows whether the group leader or the engineer should take charge for each item.  Whichever person isn’t in charge would ordinarily then keep an eye on the time and might assist, for instance, by writing things where people can see them.  You can agree on any assistance quickly as preparation before a session.

\sphinxAtStartPar
This guide describes each of the items on the summary sheets in a little more detail, giving instructions about how to run them and some tips that will help ensure everyone contributes and things run smoothly.

\sphinxAtStartPar
:TODO: Pam \sphinxhyphen{} please check what you want to say here.

\sphinxAtStartPar
We won’t have much to say in this guide about some of the smaller items, apart from describing the general pattern:
\begin{itemize}
\item {} 
\sphinxAtStartPar
\sphinxstylestrong{Icebreakers.}  These are designed to get people’s heads into the room on gently onto the topic, and to be something very brief where everyone can contribute, as a subtle reminder that even the shy or those who don’t feel they are “experts” still have an important role to play here.  It’s not important to remember what was said. It’s good practice for the group leader to model good behaviour by giving a brief answer as the first response and to keep it moving by not allowing anyone to speak at length.

\item {} 
\sphinxAtStartPar
\sphinxstylestrong{Group agreement.} In the first session, the icebreaker is followed by an explanation of the group agreement. In subsequent sessions, you only need to remind the members that the group agreement exists.

\item {} 
\sphinxAtStartPar
\sphinxstylestrong{Preparing to learn in pairs or as a whole group.}  We tend to have short items where people have a conversation to get their heads further into a topic by describing something they already know about, often to do with their home or past experience.  In the early sessions, this is done in pairs, because many people find this less daunting than speaking in front of the whole group.  In the later sessions, people have usually “warmed up” a bit.  As group leaders, you can’t hear what happens in the pairs, and that’s OK.  If the group has an odd number of people, the leader or the engineer can take a break! These items just get the group ready to deal with new concepts.

\item {} 
\sphinxAtStartPar
\sphinxstylestrong{Idea storms.}  These are whole group activities, but here the goal is to get as many ideas out as possible during the time without evaluating them.  For these, we recommend the group leader manage the group, and the engineer write down the ideas, either on a flipchart or by typing them into a word processing while projecting.   Some group members may struggle to be clear in their contributions or dominate, usually with the best of intentions \sphinxhyphen{} there are tricks you can use to make things smoother, like making it clear when there is 10 minutes/5 minutes/2 minutes to go; interrupting and summarising what you think the sense of the group is on the topic and asking other group members to confirm that; or asking if anyone who hasn’t spoken for a while has anything to say. We will cover some of these tips in the group leader training.

\item {} 
\sphinxAtStartPar
\sphinxstylestrong{Engineer talks.} These are short talks that introduce new concepts to the groups that are important for the work they will be doing during the session.  We provide a storyline for these talks and suggestions about how to give them.  There is also always a chance for group members to ask questions.

\item {} 
\sphinxAtStartPar
\sphinxstylestrong{Activities.} This is where the bulk of the groups’ exploration lies. These are specifically designed for each individual purpose but they allow the group to apply to the premises the concepts that the engineer has explained, recording the results.  Together, they build up to a proposed plan \sphinxhyphen{} the outcome of the group.

\item {} 
\sphinxAtStartPar
\sphinxstylestrong{Making decisions as a group.} Sometimes the group needs to decide on actions to be taken before the next session \sphinxhyphen{} some of these are needed as preparation, and some of them might be voluntary agreement of extra things they want to do to understand some aspect of the building and how they use it that they think might be problematic. Running these items is much like chairing regular meetings.  Some of the same tips apply as for idea storms.  If you are struggling to summarise the sense of the group, a good approach is asking if anyone else can.

\item {} 
\sphinxAtStartPar
\sphinxstylestrong{Wrap up.} It’s useful at the ends of sessions to remind group members that there is an evaluation form they can use to tell us how they found the session, including what was useful and what wasn’t.  We are happy to take this feedback any way it comes to us.  If you have sufficient time, asking this at the end could be useful to you, too.

\end{itemize}

\sphinxstepscope


\chapter{Keeping your group safe}
\label{\detokenize{safety:keeping-your-group-safe}}\label{\detokenize{safety::doc}}
\sphinxAtStartPar
Before we explain more about your roles and introduce the content for individual sessions, we want to say something about your role in keeping groups safe.  In community buildings, we often find volunteers doing things that are beyond their physical abilities, especially if they are older.  This is often because as they age, they don’t want to admit that they are no longer so strong or steady on their feet.  It’s also just part of the “can do” culture in service to the community.  We sympathise, but they can be a danger to themselves, others, and the building itself.  Before the first activity, we’ve included a short risk assessment that includes some unbendable ground rules.



\sphinxAtStartPar
Don’t forget that your leadership of the group gives you a certain status \sphinxhyphen{} please strongly reinforce our ground rules during the activities, and make more if there others appropriate to the building.

\begin{sphinxadmonition}{note}{Ground rules}

\sphinxAtStartPar
These ground rules are in the reading materials, but you’ll need to introduce them during the risk assessment and remind people of them before any activities where members might be tempted to violate them.
\begin{itemize}
\item {} 
\sphinxAtStartPar
no ladders

\item {} 
\sphinxAtStartPar
no worn steps

\item {} 
\sphinxAtStartPar
don’t go alone

\end{itemize}
\end{sphinxadmonition}

\sphinxAtStartPar
You can help reduce temptation by pointing the group towards safer alternatives.  For instance, if the meters aren’t accessible, then it is possible to check some things by looking at energy bills, and the group could arrange to get a smart meter fitted. You can also remind the group that they don’t need to gather every detail about their building.  The goal here is just to be well\sphinxhyphen{}equipped to have the right conversations with future professionals \sphinxhyphen{} who can find out the rest of what they need at the time.

\sphinxstepscope


\chapter{Dealing with complex sites}
\label{\detokenize{complex-sites:dealing-with-complex-sites}}\label{\detokenize{complex-sites::doc}}
\sphinxAtStartPar
Some of the groups in the programme have one relatively simple building, but some have multiple buildings or buildings that are split into many rooms.  For these, you won’t have time to perform activities relating to all of them, and you won’t be able to gather thermal monitoring data for all spaces, either.  Don’t be tempted to try to cover too much.  During the activities, if you need to, you can focus on one key space or one building.  This will reinforce the concepts that the group is learning.

\sphinxAtStartPar
At the end of the session, the group can agree what they want to do that relates to the other spaces.  They might want to do nothing, or repeat some exercises themselves outside sessions, or just agree what the implications are for their other spaces, depending on what time they have and what they already know about their buildings.

\sphinxstepscope


\chapter{What does the group leader do?}
\label{\detokenize{group-leadership:what-does-the-group-leader-do}}\label{\detokenize{group-leadership::doc}}
\sphinxAtStartPar
:TODO: Pam?

\sphinxAtStartPar
These are skills that will be unfamiliar to most of the group leaders, but good leadership makes it much easier to have a good group outcome.  We provide a training session for the group leaders to ensure they will be comfortable with their role.

\sphinxstepscope


\chapter{The engineer’s role}
\label{\detokenize{role-of-engineer:the-engineer-s-role}}\label{\detokenize{role-of-engineer::doc}}
\sphinxAtStartPar
In this programme, the group leader manages the “social” side of the group \sphinxhyphen{} organising sessions, introducing topics that the group needs to talk about, and liaising with the property management.

\sphinxAtStartPar
The engineers take charge of the technical exercises, some work between sessions that makes use of technical information, and asking any questions of our expert advisers or of the network of volunteers, some of whom are very experienced building services engineers. It’s useful if they also take on a little of the “social” work, for instance, keeping an eye on the time during the parts of the session they are not leading, writing down anything the group needs to all be able to see, and so on.



\sphinxAtStartPar
Some of the engineers will be operating well away from their core engineering discipline \sphinxhyphen{} and we applaud you for that.  We provide reading materials explaining and reinforcing the core concepts.  We give these to everyone involved, but it’s particularly useful for you to be comfortable with them.  There will be a chance to ask about them in the training session, but do feel free to ask us about them at any time.  The reading materials include ideas for further reading in case you are interested.  Some of the ideas are very technical.  If you are a building services engineer, you can probably improve on what we say and what we suggest.  We’re very happy for you to tell us what we ought to have done.

\sphinxAtStartPar
During the sessions, it may be that you discover issues in the building where it would be helpful to run further investigations \sphinxhyphen{} for instance, issues with heat distribution that require thinking about pumpwork or the pipework sizing and configuration in a more complicated boiler room, or checking whether a wall is insulated.  It’s useful to identify and document concerns, but this is where your responsibility stops \sphinxhyphen{} if you wish to explore further, we are happy to try to facilitate that, but please be mindful of your time so that you don’t feel overburdened.

\sphinxAtStartPar
Because you are not operating in a professional capacity, there is no contractual relationship constraining what you do and say.  Of course you will want to use your skills to help the community, but you are offering your talents just as all other community volunteers do \sphinxhyphen{} as one human to others. It’s perfectly OK to say you don’t know the answer to a question!

\begin{sphinxadmonition}{note}{Professional Indemnity}

\sphinxAtStartPar
Even if you are a building services engineer, you are not providing a professional service as part of the programme.  Neither is HeatHack.  We will stress throughout the programme that this is an exercise that should point them in the right direction, but they need to take professional advice before acting on the results.
\end{sphinxadmonition}

\sphinxAtStartPar
One of our core goals in this programme is to demonstrate how useful it is for the groups to include engineers in their community work.   Communities have talented people all around them, but because people have to move around the country to chase work opportunities, it is harder for groups to connect with them than it once was.   Engineers bring a perspective to managing community buildings that is often missing.   Thank you for being involved.

\sphinxstepscope


\chapter{Documenting the project}
\label{\detokenize{profile:documenting-the-project}}\label{\detokenize{profile::doc}}
\sphinxAtStartPar
The programme is designed to result in a profile of the building, how it is used, and future aspirations in a way that should help when writing grant proposals and agreeing briefs with heating consultants and architects.   The record can gather any useful data about the building in one place \sphinxhyphen{} photographs, architectural and heating system diagrams, documents detailing past work to reduce heat loss \sphinxhyphen{} as well as any questions arising from investigations of the heating and ventilation and the results of the group’s planning and community engagement exercise.

\sphinxAtStartPar
Where groups are comfortable enough with the technology, the easiest way is through cloud storage, for instance, using Google Drive.  The facilitator could sign up for a new Gmail address and using the space that comes with that.  It’s possible to share files on Drive with people by entering their email addresses.  Any email address will work, but the person will need to create a Google account and associate the address with the account.   It’s also possible to allow anyone with a hidden link to view the files.  If the files aren’t the least bit sensitive, they could be made public instead.  Because it’s easy to accidentally share files more openly than you intend, you should not put files with sensitive or personal details in this kind of shared filespace, especially without permission.

\sphinxAtStartPar
There are alternatives to Google Drive, such as Dropbox or Microsoft 365 Online \sphinxhyphen{} it’s a matter of what the group finds useful.  Whatever you use, it’s useful if you can share it with us so we can see it, too.

\sphinxAtStartPar
Where groups aren’t confident or happy to work in this way, you will want to find other ways to document what you can. If you think your group could use cloud storage but need us to set it up for you, please contact our administrator.

\sphinxAtStartPar
Thermal monitoring data — temperature and relative humidity measurements from the monitor that we send you — is a special case because we have specific methods for collecting it.  We describe how this works in the technology guide, including how to match it to .  If you will be moving your monitor between different locations, you’ll find it useful to keep a record of where it was, and when.  We h

\sphinxstepscope


\chapter{Expenses}
\label{\detokenize{expenses:expenses}}\label{\detokenize{expenses::doc}}
\sphinxAtStartPar
Groups are intended not to be out of pocket from engaging in the programme.  We’ve built in an average of £25 of expenses for the groups that are officially supported during our funding period.  Not all groups will need these funds, and some may need more.

\sphinxAtStartPar
The expenses are intended to cover:
\begin{itemize}
\item {} 
\sphinxAtStartPar
the cost of lithium AA batteries for the thermal monitors. Because of postal regulations, we can’t ship them with the monitors.

\item {} 
\sphinxAtStartPar
costs for the engineer to travel to the sessions \sphinxhyphen{} for instance, bus tickets or mileage rates.  Many of the engineers live within walking distance of their chosen venues, but some do not.

\item {} 
\sphinxAtStartPar
for groups in venues that don’t have things stationery items like post\sphinxhyphen{}it notes, flipcharts and pens availables, funds to buy these.

\item {} 
\sphinxAtStartPar
printing costs, although not all groups will have these and the venue might be willing to do small amounts of printing for the group.

\end{itemize}

\sphinxAtStartPar
Please contact our administrator to make arrangements if you wish to make any claims.

\sphinxstepscope


\chapter{Session 1 \sphinxhyphen{} Basic concepts}
\label{\detokenize{session1/session1:session-1-basic-concepts}}\label{\detokenize{session1/session1::doc}}
\sphinxAtStartPar
Session 1 introduces some core concepts relating to thermal comfort, thermal mass/difficulty if not heating a lot; where heat loss occurs; how cost of addressing that is unaffordable unless building has high occupancy; localised heating vs space heating; what buildings with localised heating really need is to stay dry more than anything, so you need to think about how the moisture gets in there and how to get it out

\sphinxAtStartPar
:TODO: fix introduction and whatever we need to comment about the contents (for this document).  The summary sheet is what they will actually use during the session.  Session\sphinxhyphen{}outlines.docx contains the correct format for this session although it might be too ambitious; this guide does not have the right stuff yet.  Note that the activity is just basic photographs/survey information and e.g. where the worst draughts are in their experence, not attempts to measure anything.


\section{Special preparation for the facilitator}
\label{\detokenize{session1/session1:special-preparation-for-the-facilitator}}
\sphinxAtStartPar
ask the property manager if there is any documentation of the heating system or any architectural diagrams they can use.  This could take the form of past invoices for work done, documentation or user manuals for equipment.  For churches, architectural diagrams could be in the quinquennial report or in the archives.

\sphinxAtStartPar
Flipchart/big pen, whiteboard, pack of post\sphinxhyphen{}it notes, or projector attached to a laptop with a text editor (e.g. Microsoft Word)


\section{Things for the group to bring}
\label{\detokenize{session1/session1:things-for-the-group-to-bring}}
\sphinxAtStartPar
binoculars, opera glasses; smart phones, tablets, or digital cameras.

\sphinxstepscope


\section{Engineer Talk \sphinxhyphen{} Session Concepts}
\label{\detokenize{session1/details/session-concepts:engineer-talk-session-concepts}}\label{\detokenize{session1/details/session-concepts::doc}}
\sphinxAtStartPar
This week’s engineer talk lays out some basic concepts that it is important for the group to understand, because they affect which of two major paths a community building might take in future.

\sphinxAtStartPar
:TODO: fundamentals of the talk \sphinxhyphen{} What is your building good for?

\sphinxAtStartPar
You will be expecting the programme to be about what to do to your buildings \sphinxhyphen{} and it is.  But before you consider that, you have to think about how it is designed and how you intend to use it in future.

\sphinxAtStartPar
Payback is too slow for heat loss mitigations on a building you don’t use often, and can’t get grants because the community amenity is less.

\sphinxAtStartPar
The majority of the UK’s community buildings are older, traditionally built, and in stone. They are also often not in full use.  This raises an issue.  Stone buildings are excellent at damping down the extremes in outside temperature \sphinxhyphen{} making it cooler inside during the day/in the spring and warmer inside at night/in the autumn than it would be in a more reactive building \sphinxhyphen{} but they are extremely difficult to make warm for a short time.  On the other hand, once the stones are warm, they cool down slowly.  They also are more expensive and more difficult to retrofit with heat loss mitigations, with some uncertainty about the safe ways to do things.

\sphinxAtStartPar
Thermal comfort, effects of surface temperatures, draughts, damp \sphinxhyphen{} gives us a way out \sphinxhyphen{} localised heating for the people in the space rather than the space heating we’re accustomed to.  People are pretty adaptable and avoiding draughts (worse the bigger the difference in temperature between inside and outside!) and keeping buildings dry can be the most important things.

\sphinxAtStartPar
Promises we need to keep for session 3: “Remind the group of what they learned about localised heating versus space heating, hybrid systems that, for instance, give a low level of background heat and top up with localised solutions, and also about the possibility of reconfiguring the building so it suits community needs better.”

\sphinxAtStartPar
Old list:

\sphinxAtStartPar
o Needs to convey the idea that thermal comfort relies not just on air temperatures but also the average of the surrounding surface temperatures, and the surface temperatures in a stone building that is only heated occasionally are always low for comfort.  Stone is a really good material for evening out temperatures over the day and night cycle and over time because of its thermal mass.  For church worship spaces, this is the opposite of what they want – they are coldest at around 8 am and warmest at around 6 pm.
o Single glazed window surfaces are always cold; if you’re sitting next to one that’s a lot of what influences your comfort.
o Draughts are also important (including downdraughts off cold windows, a double whammy of discomfort), and damp makes people feel worse.
o People always think of increasing the air temperature (= thermostat) to make a place more comfortable, but it’s expensive, and more expensive the higher you go.  The way most people run their homes, reducing the thermostat by a degree will save around 10\% of the energy/bill.  For most community buildings, it will be more.
o There are other ways of making people comfortable:
 More clothes!  Temperature expectations only ever go up (and there are requirements for some things, like 16C for registered child care).
 Heat the people, not the space! By making some surface near them a bit warm (under pew heaters, pew cushions) or a farther surface very warm (radiant heaters).
 Make changes to the building so it loses less heat – insulation, and so on \sphinxhyphen{} but that’s only a reasonable thing to do if you have enough users to justify a grant or other ways of paying the expense, and it’s only needed if you use space heating.
 Use the building more so you can justify putting more energy into it, making and keeping? the stone warmer and people more comfortable at the same

\sphinxstepscope


\section{Activity:  Building orientation}
\label{\detokenize{session1/details/activity:activity-building-orientation}}\label{\detokenize{session1/details/activity::doc}}\begin{itemize}
\item {} 
\sphinxAtStartPar
walk through the spaces to familiarise and discuss priorities in larger buildings where they can’t look at everything.

\item {} 
\sphinxAtStartPar
pictures of the energy meters sufficient to know how to read them and figure out what the current reading is.  Many groups don’t know how many meters they have and which show consumption for which areas.  Most groups will never look at consumption on their bills and only think in terms of cost.  If the group needs to take hand readings, smartphone photographs are a good method.  The photograph will time and date the picture automatically. How will they figure out what’s on what meter.

\item {} 
\sphinxAtStartPar
Deciding a meter reading strategy \sphinxhyphen{}  :TODO: describe what is useful \sphinxhyphen{}  if there’s good frequent readings on the bills, or access to smart meter data, OK.  Some might find it easy to take a quick photo weekly because they are passing for other reasons.  Log book at meter, meter readings app, online spreadsheet (we should set this up).

\item {} 
\sphinxAtStartPar
Deciding a thermal monitoring strategy \sphinxhyphen{} just monitor one space, or move it around and keep track of where it is?  Ideal is two weeks of data in each location in each season (acknowledge that’s not possible during the project itself.)

\end{itemize}

\begin{sphinxadmonition}{note}{Reading meters}

\sphinxAtStartPar
You may wonder why we aren’t asking groups to read their meters on the spot.  Meters can be hard to read properly, especially older ones.  It’s important to know whether gas meters are imperial, in ft3, or metric, in m3, and how to transform that into kW.  For electricity, there may be separate readings for day rates and night rates.  Without access to past bills, it may not be obvious how many meters there are, whether any are submeters, and what areas they serve.  Even just taking photos of the meters with the current readings will be an achievement for many groups.  The meter reading might be obvious when the group comes together at the end of the session, or it might be that some group member needs to take the photo away and figure out what the photo means.
\begin{itemize}
\item {} 
\sphinxAtStartPar
How to read a meter \sphinxurl{https://www.citizensadvice.org.uk/scotland/consumer/energy/energy-supply/your-energy-meter/how-to-read-your-energy-meter/}

\end{itemize}

\sphinxAtStartPar
:TODO: shift/copy link somewhere in the group materials, another group member might
be doing the reading?
\end{sphinxadmonition}

\sphinxstepscope


\section{Takeaways and Planning}
\label{\detokenize{session1/details/takeaways-and-planning:takeaways-and-planning}}\label{\detokenize{session1/details/takeaways-and-planning::doc}}\begin{itemize}
\item {} 
\sphinxAtStartPar
summarise what’s going to happen in the other sessions

\item {} 
\sphinxAtStartPar
do they want to shift the thermal monitor around some spaces (two weeks at each), or leave it in one?  If they move it, they’ll want to fill in the Thermal Monitor Location Diary:
\begin{itemize}
\item {} 
\sphinxAtStartPar
\sphinxhref{https://docs.google.com/spreadsheets/d/1Lb59luV7bnODQef9KC9vKmHjVDsIbQYyRfcX4VaVAA4/}{Google Docs \sphinxhyphen{} Monitor Location Diary Template}

\end{itemize}

\end{itemize}


\begin{itemize}
\item {} 
\sphinxAtStartPar
any existing description of who uses the building and the group’s values? If it’s a church and they’ve searched for a new rector, vicar, or minister recently, they may have something useful about the community and how the building is used from the profile they constructed for that.  If not, often on the website

\item {} 
\sphinxAtStartPar
heating system documentation and floorplans, for second session.

\item {} 
\sphinxAtStartPar
arranging to find annual energy consumption from past energy bills \sphinxhyphen{} if a church or otherwise part of a larger organisation, someone may already have calculated a year’s gas and electricity consumption as part of wider net zero reporting.  I think there’s always a summary of annual consumption because suppliers are required by law to give historical information if customers ask (The Electricity and Gas (Billing) Regulations 2014)  It’s a pain if all the readings are estimated.

\end{itemize}

\sphinxAtStartPar
HINT:  for churches, there might be someone in the “green group” or on the property committee who has already done a carbon footprint baseline that required them to get information from past energy bills.
If there is a smart meter, the utility company might give access to the past readings, or at least provide some analysis of them.
\begin{itemize}
\item {} 
\sphinxAtStartPar
:TODO: is going to Canmore ever useful? probably not.

\end{itemize}

\sphinxstepscope


\section{The Engineer’s Report}
\label{\detokenize{session1/details/engineers-report:the-engineer-s-report}}\label{\detokenize{session1/details/engineers-report::doc}}
\sphinxAtStartPar
brief report about the building, if no better documentation exists

\sphinxAtStartPar
:TODO: give them an example from CCM

\sphinxAtStartPar
:TODO: create a template for it?  But AM’s is just free text.  Maybe a ticklist of things to describe:
\begin{itemize}
\item {} 
\sphinxAtStartPar
age, number of buildings

\end{itemize}

\sphinxstepscope


\chapter{Session 2}
\label{\detokenize{session2/session2:session-2}}\label{\detokenize{session2/session2::doc}}
\sphinxstepscope


\section{Activity:  Building survey}
\label{\detokenize{session2/details/activity:activity-building-survey}}\label{\detokenize{session2/details/activity::doc}}
\sphinxAtStartPar
This activity is a simplified version of what energy efficiency consultants do during a site survey. During it, the group will fill out a form with basic information about the building and take photographs that relate to heating, heat loss, ventilation.

\sphinxAtStartPar
There are two purposes to this activity.

\sphinxAtStartPar
The first is to familiarise the group with aspects of the building they may never have considered.  Many groups will have members who may only have hazy memories of some spaces and, for instance, never have looked at the windows.  The building might also be completely new to the engineer, although we expect some engineers will have been given a tour when they first matched with a group.

\sphinxAtStartPar
The second is as a memory aid and for some groups, to serve as the only record of some aspects of the building’s heating, ventilation, and heat loss features.  Community buildings often lack any documentation of the heating system, for instance.  Even if there is a diagram from the original installer, it will fail to convey its current state.  Surveys often make use of photography to highlight issues and both heating professionals and architects have told us that having them available would help them use their time more effectively.  Pictures may help groups and engineers help each other and also allow us to ask some questions of our Energy Adviser, Andrew Macowan.

\sphinxAtStartPar
:TODO: somewhere put that group leaders should try to get any heating system documentation and floorplans in time for the second session if they can \sphinxhyphen{} this could be mentioned in the group leader training.

\sphinxAtStartPar
:TODO: create the form based on Andrew’s and maybe some aspects of the CSE one.

\sphinxAtStartPar
In this activity, the group will take photographs that relate to heating, heat loss, ventilation, and energy use.  They will do this either for an entire building, if it is simple, or one of its larger spaces if it is not.  As they work .  They should divide into pairs, but if not enough people have smartphones or cameras, whatever number of groups is possible. How to assign the pairs depends on the building \sphinxhyphen{} for instance, ask for pictures of the meters from whoever will be going closest to them.  If the gas meter is outside, that will mean whoever goes outside to look for ventilation; if it is in or near the boiler room, that means whoever is documenting the heating system.

\sphinxAtStartPar
:TODO: get the rest of this right.

\sphinxAtStartPar
There are six main things you are looking from the activity:
\begin{itemize}
\item {} 
\sphinxAtStartPar
representative pictures of the heat output devices in the space.  Many groups have basic misconceptions about how radiators and other devices work.  This leads to them blocking the air flow on radiators or turning off the fans on fan convectors.  It can also lead to them trying to preheat a space with devices that are only meant to provide localised heating for the comfort of the occupants.

\item {} 
\sphinxAtStartPar
pictures of the heating controls in the space, including on electrical appliances.  Many groups don’t know if they have a thermostat, whether it is wired in to the current system, and whether it is effective in practice.  Many controls that users can access will allow them to do the wrong things, like turn the heating on (or off) permanently, turn it up to an unaffordable level, or turn on the heating for a system where there’s no chance it will improve their comfort for the time they’ll be in the space.  Thermostatic radiator valves in community spaces cause real issues for thermal comfort, energy waste, or both.

\item {} 
\sphinxAtStartPar
representative pictures of the glazing.

\item {} 
\sphinxAtStartPar
representative pictures of the ventilation features of the building, including outside the building.  They are unlikely to get these right, and some may be in areas they can’t access, but this is to get some idea and get them thinking about ventilation as they go about the rest of their business in the building between sessions.

\item {} 
\sphinxAtStartPar
pictures of the boiler room, if there is one.  These are really for us, to give us some idea of what you’re dealing with.    If you are a building services engineer, you may not need assistance with this.  Many community buildings have had single boilers replaced with multi\sphinxhyphen{}boiler installations starting in the 1990s, and often these have been plumbed in incorrectly or are under insufficient control.  Some corrections are too expensive to consider until we understand how much future heating will rely on hydrogen \sphinxhyphen{} a decision expected to come in 2026 \sphinxhyphen{} but it’s useful to understand the situation as it will affect decisions about how to stage future changes that rely on “hybrid” heating approaches and what can be said in grant applications.  It’s easier to make the case for replacing non\sphinxhyphen{}condensing boilers than condensing ones that recover heat from the flue.

\item {} 
\sphinxAtStartPar
This is also a good time to check that your processes for meter reading are as good as you can achieve.

\end{itemize}

\sphinxstepscope


\section{Takeaways and planning}
\label{\detokenize{session2/details/takeaways-and-planning:takeaways-and-planning}}\label{\detokenize{session2/details/takeaways-and-planning::doc}}
\sphinxAtStartPar
It’s useful to gather the photos for the building profile if you can \sphinxhyphen{} at least anything you are uncertain about or that you can’t easily describe in text.  You will need to determine how you will collect them together to put them on your cloud storage.

\sphinxAtStartPar
For the next session, you’ll also need either the actual diary showing who is in the spaces for a period where you have collected thermal monitoring data, or a diary that shows typical building use over the course of week, for instance, in this format:
\begin{itemize}
\item {} 
\sphinxAtStartPar
\sphinxhref{https://docs.google.com/spreadsheets/d/1\_3UwlKGqtnaVQqrsQDyNMr6MdldH\_sSLpiHTBwC7AbQ/}{Google Docs \sphinxhyphen{} Building Use Diary Template}

\end{itemize}

\sphinxAtStartPar
It’s helpful if you can pass the engineer this information ahead of the session.



\sphinxstepscope


\chapter{Session 3}
\label{\detokenize{session3/session3:session-3}}\label{\detokenize{session3/session3::doc}}
\sphinxAtStartPar
Engineer ahead looks at the traces just to get organised about what to ask about \sphinxhyphen{} but only the group will know about how the building is used, what’s likely to be wrong.

\sphinxstepscope


\section{Session Preparation}
\label{\detokenize{session3/details/preparation:session-preparation}}\label{\detokenize{session3/details/preparation::doc}}

\section{What to bring}
\label{\detokenize{session3/details/preparation:what-to-bring}}
\sphinxAtStartPar
Flipchart (or use projection)


\section{Group leader}
\label{\detokenize{session3/details/preparation:group-leader}}\begin{itemize}
\item {} 
\sphinxAtStartPar
Large rough floorplan of the building on at least A1 paper (e.g. flipchart paper).

\end{itemize}


\section{Engineer}
\label{\detokenize{session3/details/preparation:engineer}}
\sphinxAtStartPar
It’s helpful if the engineer can check for some common heating problems in the traces ahead of the session \sphinxhyphen{} for instance, the heating unexpectedly being on at night or for long periods.  We will provide better instructions for this with plots that help spot them in October.

\sphinxstepscope


\section{Engineer Talk}
\label{\detokenize{session3/details/session-concepts:engineer-talk}}\label{\detokenize{session3/details/session-concepts::doc}}
\sphinxstepscope


\section{Session Activity \sphinxhyphen{} Mapping Building Use}
\label{\detokenize{session3/details/activity:session-activity-mapping-building-use}}\label{\detokenize{session3/details/activity::doc}}
\sphinxAtStartPar
This activity works through how the building is used on typical days to help groups think through what experiences are like for their current users, and give them a structure for thinking about what building changes might make under\sphinxhyphen{}used spaces more useful for any unmet needs in the local community.  We have prepared a video showing the activity because it’s easier than it sounds when we write it down!

\sphinxAtStartPar
:TODO:  Insert video link and a still image.
\begin{itemize}
\item {} 
\sphinxAtStartPar
\DUrole{xref,myst}{Youtube Video \sphinxhyphen{} Session 3 Activity}

\end{itemize}




\subsection{What you need for the activity}
\label{\detokenize{session3/details/activity:what-you-need-for-the-activity}}
\sphinxAtStartPar
For this activity, you will need:
\begin{itemize}
\item {} 
\sphinxAtStartPar
a rough floorplan of the building drawn out large enough for everyone to see, for instance, on a piece of A1 flipchart paper, or perhaps two taped together.   It doesn’t need to be beautiful \sphinxhyphen{} it just needs your spaces on it with entrances and exits.

\item {} 
\sphinxAtStartPar
either chess pieces or coins in a variety of denominations.

\item {} 
\sphinxAtStartPar
the diary that shows what groups are in the building when over a typical week.  If your building use is either very regular or it varies greatly from week to week, you may want to use the actual diary.  Many groups use Google Calendar for this \sphinxhyphen{} you may be able to arrange for one person to have access to that or to receive an exported copy.  Be careful about any personal or sensitive details that these diaries can reveal.  For this activity, those aren’t needed, but it is useful to know something about the sizes of the groups and whether the people involved have special heating needs, for instance, because they are children, old, or infirm.  We have a diary template that can be used to extract this information so it is easy to use during the session \sphinxhyphen{} whether you need it depends on how complicated your building use is.
\begin{itemize}
\item {} 
\sphinxAtStartPar
\sphinxhref{https://docs.google.com/spreadsheets/d/1\_3UwlKGqtnaVQqrsQDyNMr6MdldH\_sSLpiHTBwC7AbQ/}{Google Docs \sphinxhyphen{} Building Use Diary Template}

\end{itemize}

\end{itemize}


\begin{itemize}
\item {} 
\sphinxAtStartPar
access to the thermal monitoring data.  This is available here :TODO: :
\begin{itemize}
\item {} 
\sphinxAtStartPar
\sphinxhref{https://jeancarletta.github.io/HeatHack-Data/}{Github Pages \sphinxhyphen{} Thermal Monitoring Results}

\end{itemize}

\item {} 
\sphinxAtStartPar
If you have been moving the monitor to different places in your buildings, they will also find it useful to have access to the Monitor Location Diary that describes where it was when.  Each group stores this separately but it looks like the template here:
\begin{itemize}
\item {} 
\sphinxAtStartPar
\sphinxhref{https://docs.google.com/spreadsheets/d/1Lb59luV7bnODQef9KC9vKmHjVDsIbQYyRfcX4VaVAA4/}{Google Docs \sphinxhyphen{} Monitor Location Diary Template}

\end{itemize}

\end{itemize}


\begin{itemize}
\item {} 
\sphinxAtStartPar
the three discussion questions written out for the group leader to remember, or preferable, written out large (or projected) where everyone will be able to see them.

\end{itemize}


\subsection{How to run the activity}
\label{\detokenize{session3/details/activity:how-to-run-the-activity}}
\sphinxAtStartPar
For the activity, you use coins or chess pieces to enact a typical week in the life of the building.  Small value coins (or pawns) can represent single people or very small groups, mid\sphinxhyphen{}value coins (or e.g., knights and rooks) can represent medium\sphinxhyphen{}sized groups, and pound coins (or kings and queens) can represent large groups.  Start, for instance, at 8:00 on a Monday morning and step through what happens in the building up until lunchtime, walking groups through the building. What entrances do they use?  Where do they go?

\sphinxAtStartPar
At lunchtime, early evening, and building closure  — or whatever points make the most sense for your circumstances — review the following questions:
\begin{enumerate}
\sphinxsetlistlabels{\arabic}{enumi}{enumii}{}{.}%
\item {} 
\sphinxAtStartPar
What temperature do they groups need their spaces to be, and are they getting them?  Does the temperature data show anything undesirable, for instance, unexplained changes or large fluctuations (over 2C) that would make people uncomfortable? What can/do they do if they are too hot or too cold?

\item {} 
\sphinxAtStartPar
Are we heating lots of spaces that aren’t being used just because they are on the same heating systems, and is that something we can correct?  How would we do that?

\item {} 
\sphinxAtStartPar
If the building isn’t in full use, what’s blocking that?  Is it lack of need in the community?  Do we know what unmet needs there might be at this time?  Are our spaces too big?  Too small? Have the wrong type of flooring and seating, or the wrong facilities?

\end{enumerate}

\sphinxAtStartPar
After considering enough different times in the week to make sense of your building’s use, it’s time to think about the big picture.  Remind the group of what they learned about localised heating versus space heating, hybrid systems that, for instance, give a low level of background heat and top up with localised solutions, and also about the possibility of reconfiguring the building so it suits community needs better.  Go around the room for people to suggest one thing they learned from the exercise that they think should influence future choices, or one action that the group should seriously consider.  Record ideas on a flipchart (or projected) and let them flow into a wider discussion, but ensure that everyone gets a chance to contribute, as the best understanding won’t always be from the people most inclined to speak up.  You can then photograph the flipchart to remember what was said.  We will be revisiting these possible future actions in session 4.

\sphinxAtStartPar
During the exercise, depending on their experience, the engineer may have instincts that some things as not worth considering as they arise.  For instance, groups might be tempted to turn off the heating in all rooms but the one being used \sphinxhyphen{} that can work, but sometimes it has such a big effect on the building that users can’t be made comfortable.  A structural engineer or an engineer who works with architects might have very good instincts about whether opening spaces out, creating mezzanines, and so on is likely.  The engineer should guide the group where they feel comfortable to make the best use of the group’s time, but unlike in a professional setting, it’s absolutely fine to not know the answer to a question or to be uncertain about the answer.

\sphinxstepscope


\section{Takeaways and Planning}
\label{\detokenize{session3/details/takeaways-and-planning:takeaways-and-planning}}\label{\detokenize{session3/details/takeaways-and-planning::doc}}
\sphinxstepscope


\chapter{Session 4}
\label{\detokenize{session4/session4:session-4}}\label{\detokenize{session4/session4::doc}}
\sphinxAtStartPar
:TODO: write \sphinxhyphen{} sketch below


\section{Consolidate:  use PATH to plan what changes they want to make in the building}
\label{\detokenize{session4/session4:consolidate-use-path-to-plan-what-changes-they-want-to-make-in-the-building}}

\section{Brainstorming actions}
\label{\detokenize{session4/session4:brainstorming-actions}}
\sphinxAtStartPar
Priortise them a, b, c, etc

\sphinxAtStartPar
Use the A’s as input to the PATH


\section{Plan community engagement event}
\label{\detokenize{session4/session4:plan-community-engagement-event}}
\sphinxAtStartPar
Input as to how this event can happen and what is the desired outcome (

\sphinxAtStartPar
The group decides:

\sphinxAtStartPar
what is your desired outcome?

\sphinxAtStartPar
Date, time, duration of event, where

\sphinxAtStartPar
How to raise awareness before invite goes out, eg minister, eg newsletter (input as to how this can happen)

\sphinxAtStartPar
Who to invite

\sphinxAtStartPar
Use Spheres of Influence exercise to increase list

\sphinxAtStartPar
Who does what by when for this engagement event plan

\sphinxAtStartPar
Esp Who will do the ‘writing up’ of this process including the Community Engagement Event outcomes, what format wil this be in?  formal report?  Video? PATH plan plus any outputs of Community Engagment Events


\section{What is next for this group – closing}
\label{\detokenize{session4/session4:what-is-next-for-this-group-closing}}
\sphinxAtStartPar
We’ve worked hard together etc, is there a life for this group beyond these sessions??

\sphinxAtStartPar
Proper closing of session, noting that the group needs to stay in touch for the event planning and implementation

\sphinxstepscope


\section{Session Preparation}
\label{\detokenize{session4/details/preparation:session-preparation}}\label{\detokenize{session4/details/preparation::doc}}
\sphinxstepscope


\section{Engineer Talk}
\label{\detokenize{session4/details/session-concepts:engineer-talk}}\label{\detokenize{session4/details/session-concepts::doc}}
\sphinxstepscope


\section{Activity}
\label{\detokenize{session4/details/activity:activity}}\label{\detokenize{session4/details/activity::doc}}
\sphinxstepscope


\section{Takeaways and Planning}
\label{\detokenize{session4/details/takeaways-and-planning:takeaways-and-planning}}\label{\detokenize{session4/details/takeaways-and-planning::doc}}






\renewcommand{\indexname}{Index}
\printindex
\end{document}